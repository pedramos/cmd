%% start of file `template.tex'.
%% Copyright 2006-2015 Xavier Danaux (xdanaux@gmail.com), 2020-2022 moderncv maintainers (github.com/moderncv).
%
% This work may be distributed and/or modified under the
% conditions of the LaTeX Project Public License version 1.3c,
% available at http://www.latex-project.org/lppl/.


%----------------------------------------------------------------------------------------
%	PACKAGES AND OTHER DOCUMENT CONFIGURATIONS
%----------------------------------------------------------------------------------------

\documentclass[12pt,a4paper, roman]{moderncv} % Font sizes: 10, 11, or 12; paper sizes: a4paper, letterpaper, a5paper, legalpaper, executivepaper or landscape; font families: sans or roman

\usepackage[portuguese]{babel} 
\usepackage[utf8]{inputenc} 

\moderncvstyle{classic} % CV theme - options include: 'casual' (default), 'classic', 'oldstyle' and 'banking'
\moderncvcolor{grey} % CV color - options include: 'blue' (default), 'orange', 'green', 'red', 'purple', 'grey' and 'black'

\usepackage[scale=0.75]{geometry} % Reduce document margins
%\setlength{\hintscolumnwidth}{3cm} % Uncomment to change the width of the dates column

% For the 'classic' style, uncomment to adjust the width of the space allocated to your name
%\setlength{\makecvtitlenamewidth}{10cm} 

%\usepackage[defaultsans]{lato}
%\renewcommand\familydefault{\sfdefault}{\lato}

\usepackage[T1]{fontenc}
\usepackage[sfdefault]{AlegreyaSans} %% Option 'black' gives heavier bold face
%% The 'sfdefault' option to make the base font sans serif
\renewcommand*\oldstylenums[1]{{\AlegreyaSansOsF #1}}

\usepackage[charter]{mathdesign}
\renewcommand{\rmdefault}{mdbch}

\renewcommand*\namefont{\sffamily \LARGE \fontseries{l}\selectfont }
\renewcommand*\titlefont{\sffamily \LARGE \fontseries{l}\selectfont }

\renewcommand*\sectionfont{\sffamily \Large \fontseries{l}\selectfont }
\renewcommand*\subsectionfont{\sffamily \large \fontseries{l}\selectfont }



\usepackage[T1]{fontenc}



%----------------------------------------------------------------------------------------
%	NAME AND CONTACT INFORMATION SECTION
%----------------------------------------------------------------------------------------

\firstname{Pedro} % Your first name
\familyname{Lorga Ramos} % Your last name

% All information in this block is optional, comment out any lines you don't need
\title{Curriculum Vitae}
\address{Rua Professor Francisco Varela da Costa, n24}{7005-586 ÉVORA}
\mobile{967322766}
%\phone{(000) 111 1112}
%\fax{(000) 111 1113}
\email{pedrolorgaramos@tecnico.ulisboa.pt}
%\homepage{staff.org.edu/~jsmith}{staff.org.edu/$\sim$jsmith} % The first argument is the url for the clickable link, the second argument is the url displayed in the template - this allows special characters to be displayed such as the tilde in this example
%\extrainfo{Remote only}

% Social icons
\social[linkedin]{pedro-lorga-ramos}                        % optional, remove / comment the line if not wanted
% 
\social[github]{pedramos}                              % optional, remove / comment the line if not wanted
% \social[gitlab]{jdoe}                              % optional, remove / comment the line if not wanted
% \social[codeberg]{jdoe}                            % optional, remove / comment the line if not wanted
% \social[bitbucket]{jdoe}                           % optional, remove / comment the line if not wanted
% \social[stackoverflow]{0000000/johndoe}            % optional, remove / comment the line if not wanted
% 
% \social[skype]{jdoe}                               % optional, remove / comment the line if not wanted
% \social[orcid]{0000-0000-000-000}                  % optional, remove / comment the line if not wanted
% \social[researchgate]{jdoe}                        % optional, remove / comment the line if not wanted
% \social[researcherid]{jdoe}                        % optional, remove / comment the line if not wanted
% \social[googlescholar]{googlescholarid}            % optional, remove / comment the line if not wanted
% 
% \social[twitter]{ji\_doe}                          % optional, remove / comment the line if not wanted
% \social[mastodon]{mastodon.social/web/@user}       % optional, remove / comment the line if not wanted
% \social[telegram]{jdoe}                            % optional, remove / comment the line if not wanted
% \social[whatsapp]{12345678901}                     % optional, remove / comment the line if not wanted
% \social[signal]{12345678901}                       % optional, remove / comment the line if not wanted
% \social[matrix]{@johndoe:matrix.org}               % optional, remove / comment the line if not wanted
% \social[discord]{jdoe\#0000}                       % optional, remove / comment the line if not wanted
% 


\photo[70pt][0.4pt]{pictures/cvphoto} % The first bracket is the picture height, the second is the thickness of the frame around the picture (0pt for no frame)
%\quote{"A witty and playful quotation" - John Smith}

%----------------------------------------------------------------------------------------

\begin{document}

\makecvtitle % Print the CV title

%----------------------------------------------------------------------------------------
%	EDUCATION SECTION
%----------------------------------------------------------------------------------------

\section{Personal Information}
\cventry{Name}{Pedro Miguel Vieira Lorga Ramos}{}{}{}{}
\cventry{Birth Date}{6 of November of 1990}{}{}{}{}
\cventry{Address}{Rua Professor Francisco Varela da Costa, n24}{7005-586 ÉVORA}{}{}{}{}
\cventry{Email and cellphone}{pedrolorgaramos@tecnico.ulisboa.pt, 967322766}{}{}{}{}
\cventry{Work location}{Remote Only}{}{}{}{}

\section{Language}

\cvitemwithcomment{Portuguese}{Native Language}{}
\cvitemwithcomment{English}{Advanced}{}

%----------------------------------------------------------------------------------------
%	WORK EXPERIENCE SECTION
%----------------------------------------------------------------------------------------

\section{Experience}

\cventry{2018-current}
	{  R\&D Support and Maintenance Team}
	{Nokia Networks}{Lisbon - Portugal}{}
	{
% 		During this period I have been working as a member of Support and Maintenance team inside Nokia's R\&D
% 		developing two Nokia products. The products being develop in my R\&D department were Nokia Performance
% 		Manager, a system which collects and presents all measurements from the network, and Service Quality
% 		Manager, which is a system which can process the measurements of the network and raise alarms and sugest
% 		actions to improve the quality of the services being delivered by the telecommunications provider.
% 		Inside my R\&D department, as a member of the Maintenance and Support team, my main responsabilities
% 		are: delivering trainning reagading installation and troubleshooting of Nokia products, providing
% 		advanced services to solve specific customer needs, providing support to Support teams while handling
% 		maintenance tickets, installation of pilot software. All of the tasks meant to be executed in direct
% 		contact with the developers to keep product's final quality in check. Out of these, I focused developing
% 		tools to apply advanced services, and also applying the advanced services. These services included large
% 		database updates, needed due to inconsistent data among different tables, and system's performance
% 		assessment, based on a large amount of performance indicators automatically collected, and graphs
% 		automatically generated by 'R' scripts which are automatically layed out in docx format.
		Responsabilities in this role include:
		\itemize {
			\item {
				Providing help to Nokia Support teams on troubleshooting problems found on live systems
				which they could not solve
			}
			\item {Find and manage faults on the system and guarantee R\&D produces proper fixes to such faults.}
			\item {
					Develop and execute advanced services, like Performance Assessments,
					system fine tuning, database inconsistency fixes and test new versions before their release.
			}
		}
}



\cventry{2014-2018}{Case Handling Enginner (CSE) // Implementation Engineer}
	{Nokia Networks}
	{Lisbon - Portugal}{}
	{
% 		 CSE is responsible for the first attempt of solving the problems presented by customers, direct
% 		 communication to customer at all times and escalate to the next level of support if required
% 		 (collecting all necessary information and logs required by the next level of support). Also performed
% 		 System Acceptance Tests and implementations tasks for NetAct and later VoLTE deployment projects in
% 		 Proximus (Belgium) per specific request from Nokia's Belgium team. Later I was also selected to join L3
% 		 team (team which provides third level of support, CSE is second level) in the scope of a ticket backlog
% 		 reduction program. I provided support for the following products:
% 		  Nokia Profile Server, MMS Center, SMS
% 		 Center, Flexi Content Optimizer, messaging products, and NetAct, Nokia's OSS solution and NetAct
% 		 Advanced Configurator.
		During this period I have worked in the position of CSE, but due to my influence with the customer I
		ended up doing other tasks as implementation engineer and system administration for a specific customer
		only. . During this period I've worked with mostly with Nokia's OSS solution, NetAct. My
		responsabilities were:
		\itemize {
			\item {
				Direct contact with customer at all times to troubleshoot problems in the system
				identified by team, or to provide support while using the product.
			}
			\item {
				Identify problems with the product and route the issue to R\&D so that the fault can be
				quickly fixed.
			}
			\item {
				Work along customers' engineering team to maintain, configure and help on implementing
				large projects. One example is the introduction of new technologies (VoLTE) or large
				upgrades on core elements of customer's network.
			}
		}
}



\cventry{2011-2013}
	{Volunteer at Electrical and Computer Engineering Week in Instituto Superior Técnico}
	{Jornadas De Engenharia Electrotécnica e Computadores (JEEC 12 e JEEC 13)}
	{Lisbon - Portugal}{}
	{
		Inside telecommunications team, I helped in the decisions related to the
		selection of the speakers in JEEC and I participated in the communication between the team and the
		desired speaker
	}


%----------------------------------------------------------------------------------------
%	COMPUTER SKILLS SECTION
%----------------------------------------------------------------------------------------

\section{Skills}
\subsection{Computer Engineering}

\cvitem{Operating Systems}{Linux Administration \& Performance Assessment}
\cvitem{Databases}{Oracle DB, MariaDB}
\cvitem{Cloud}{VMWare Vsphere, Openstack, Kubernetes, Docker}

\subsection{Programing Languages}
\cvitem{Basic}
	{ HTML, CSS, Java, C++}
\cvitem{Medium}
	{Perl}
\cvitem{Advanced}
	{SQL, Python, C, sh, awk, Go, R}

\subsection{Network Elements and Network Management}
\cvitem{Advanced}{OSS (Nokia's NetAct),Nokia Profile Server, MMS Center, SMS Center, Flexi Content Optimizer, Nokia Browsing Gateway, Nokia Performance Manager, Nokia Assurance Center}

% \cvskillhead[-0.1em]
% \cvskillentry*{Language:}{5}{SQL}{10}{Most experienced with OracleDB}
% \cvskillentry*{Language:}{4}{C}{3}{}
% \cvskillentry*{Language:}{5}{Go}{4}{}
% \cvskillentry*{Language:}{2}{Perl}{2}{}
% \cvskillentry*{Language:}{4}{R}{4}{}
% \cvskillentry*{Language:}{4}{Python}{5}{}


%----------------------------------------------------------------------------------------
%	LANGUAGES SECTION
%----------------------------------------------------------------------------------------

\section{Education}

\cventry{2008-2013}
	{Masters Degree in Electrical and Computer Engineering}
	{Universidade Técnica de Lisboa - Instituto Superior Técnico}
	{Main speciality in telecommunications and secondary speciality in computers }
	{Final grade of 14,3}{} 

\section{Publications}
\subsection{Master Thesis}{}
\cvitem{Title}{Text Driven Forecasting}
\cvitem{Supervisors}{Professor Mário Figueiredo \& André Martins}
\cvitem{Description}{
	The goal of text-driven forecasting is to build a system that is able to predict numeric quantities given a body
	of text in natural language. Examples are: predicting the revenue of movies from Twitter posts, predicting
	opinion polls from blogs, predicting stock volatility given financial reports, predicting the number of external
	links given a news article, etc. The goal of this project is to apply machine learning techniques, such as
	regression, for this task.
}
\cvitem{Document}{https://fenix.tecnico.ulisboa.pt/cursos/meec/dissertacao/2353642464266}




%----------------------------------------------------------------------------------------
%	COVER LETTER
%----------------------------------------------------------------------------------------

% To remove the cover letter, comment out this entire block

%\clearpage
% 
%\recipient{HR Departmnet}{Nokia-Siemens\\Rua Irmãos Siemens 1-1-A\\2720-093 Amadora, Portugal} % Letter recipient
%\date{\today} % Letter date
%\opening{Hello,} % Opening greeting
%\closing{Best Regards} % Closing phrase
%\enclosure[Attached]{curriculum vit\ae{}} % List of enclosed documents
%
%\makelettertitle % Print letter title
%
%blah blah blah
%
%\makeletterclosing % Print letter signature

%----------------------------------------------------------------------------------------

\end{document}
